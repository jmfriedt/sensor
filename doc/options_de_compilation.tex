\documentclass[a4paper]{article}
\usepackage[french]{babel}
\usepackage{fullpage,graphicx}
\begin{document}
\begin{center}
{\Large\bf Compilation du firmware interrogateurs sans fil} \\
\today, J.-M Friedt
\end{center}

\section{Versions de lecteurs}

Le logiciel embarqu\'e est d\'esormais compatible avec 3 types d'interrogateurs afin
de mutualiser la partie algorithmique du code ({\tt DDS.c}) et de ne faire appel
qu'\`a quelques fonctions sp\'ecfiques au mat\'eriel de chaque plateforme mat\'erielle~:
\begin{enumerate}
\item les lecteurs classique et Schrader exploitent le microcontr\^oleur ADuC7026 et le DDS AD9954. Une
variante est la PLL du lecteur classique qui a \'et\'e modifi\'ee en novembre 2012 mais n'impose qu'une option 
additionnelle,
\item le lecteur gilloux bas\'e sur un DDS AD9958 et un microcontr\^oleur STM32 Cortex M3,
\item le lecteur low cost bas\'e sur un STM32 Cortex M3 et un radiomodem Semtech XE1203F
\end{enumerate}

\section{Variables d'environnement et outils de compilation}

Il est peu judicieux de d\'eplacer des versions obsol\`etes de {\tt libstm32} avec les divers
codes sources de logiciels embarqu\'es. Le choix a \'et\'e fait, en vue de permettre l'utilisation
de fonctions gourmandes en ressources que sont l'allocation dynamique de m\'emoire ({\tt malloc()})
et le calcul en virgule flottante ({\tt double}), de se lier avec la derni\`ere version de {\tt libstm32}
disponible sur {\tt summon-arm-toolchain}. Pour ce faire, les variables d'environnement suivantes sont
n\'ecessaires

\begin{verbatim}
export CORTEX_BASE_DIR=${HOME}/sat/
export STM32_INC=${CORTEX_BASE_DIR}/include
export STM32_LIB=${CORTEX_BASE_DIR}/lib
export PATH=${HOME}/sat/bin:$PATH
\end{verbatim}

\section{{\tt Makefile}}

D'un point de vue compilateur, le choix de la cha\^\i ne de compilation et des options pour le
processeur associ\'e sont d\'efinis dans {\tt Makefile}. L'utilisateur n'a, {\em a priori}, jamais
\`a toucher aux configurations de compilation incluses ({\tt makefile.*}). Trois options sont
disponibles dans {\tt Makefile} pour les trois architectures cit\'ees ci-dessus~:
\begin{enumerate}
\item {\tt include makefile.aduc} pour les lecteurs classique et Schrader,
\item {\tt include makefile.cortex\_AD9958} pour le lecteur gilloux,
\item {\tt include makefile.cortex\_XE1203} pour le lecteur low cost,
\item {\tt include makefile.pc} est une option additionnelle qui permet de tester sur PC (ie sans 
mat\'eriel embarqu\'e d\'edi\'e) les algorithmes de commande. L'int\'er\^et de cette approche
est de pouvoir synth\'etiser des signaux respectant certaines statistiques de bruit et de
valider par des tests automatiques le fonctionnement de la partie algorithmique du firmware ({\tt DDS.c}).
Ces signaux sont synth\'etis\'es dans {\tt dummy\_pc.c}, dans la fonction interroge qui renvoie un
puissance fonction de la fr\'equence sond\'ee (faisant appel aux registres FTW0, cette m\'ethode de
d\'eveloppement n'est pas fonctionnelle pour le moment pour le low cost et n'a \'et\'e test\'ee que
pour le lecteur classique). Le binaire r\'esultant, {\tt interrogateur}, est en g\'en\'eral ex\'ecut\'e
dans la console en mode texte, l'affichage dans un {\tt xterm} induisant un saut de ligne entre
les trames.
\end{enumerate}

Une seule de ces options ne peut \^etre active lors d'une compilation donn\'ee, les autres options
doivent \^etre comment\'ees en pr\'ec\'dent la ligne d'un symbole \#.

Par ailleurs, deux variables de compilation sont d\'efinies (=1) ou non (=0) dans le {\tt Makefile}
pour activer l'affichage sur \'ecran LCD (uniquement test\'e sur lecteur Schrader) ou pour
activer la communication sur bus MOSBUS (uniquement test\'e sur interrogateurs classique et Schrader
pour une plateforme autour de l'ADuC7026 -- le portage au Cortex M3 n'a pas \'et\'e valid\'ee dans tous
les cas et notamment si le lecteur est serveur). Ces options sont par d\'efaut {\tt MODBUS = 0} et
{\tt LCD    = 0}.

\section{Configurations}

La configuration logicielle doit refl\'eter les besoins du mat\'eriel. Nous d\'eclinons de nouveau
les options sp\'ecifiques \`a chaque mat\'eriel dans les options propos\'ees dans le fichier
{\tt DDSinc.h} qui d\'epend des architectures~:
\begin{enumerate}
\item pour les lecteurs classique et Schrader, le c\oe eur du DDS est cadenc\'e \`a 200~MHz ({\tt \#undef f400}),
la PLL est s\'electionn\'ee en fonction de la date de fabrication du lecteur ({\tt \#undef ADF4111} pour
pr\'e-Novembre 2012, {\tt \#define ADF4111} sinon) et le DDS est un AD9954 (donc {\tt \#undef AD9958}),
\item pour le lecteur gilloux, le c\oe eur du DDS est cadenc\'e \`a 360~MHz ({\tt \#define f400}), il
n'y a pas de PLL donc l'option {\tt ADF4111} est sans importance, et le DDS est un AD9958 (donc {\tt \#define AD9958}),
\item pour le low cost, malgr\'e l'absence de DDS nous consid\'erons qu'il est peu judicieux d'activer l'option
{\tt AD9958} et dans le doute afficherons un message d'erreur si cette configuration est active. Par ailleurs, la
fr\'equence est suppos\'ee d\'efinie comme si une horloge caden\c cant le DDS \`a 200~MHz existait, donc il
faut {\tt \#undef f400}.
\end{enumerate}

Le fichier {\tt DDSinc.h} d\'efinit les options de compilation tandis que les variables susceptibles d'\^etre
modifi\'ees par l'utilisateur (dans la fonction {\tt communication()} de {\tt DDS.c}) sont initialis\'ees avec
des valeurs par d\'efaut dans {\tt DDSvar.c}. On y trouvera donc les bornes de fr\'equences, nombre de r\'esonances
et nombre de points de mesure, configuration des fr\'equences par intervalles de tailles \'egales ou manuels ...

On pr\'ef\`erera d\'esactiver la d\'efinition d'une constante en la commentant plut\^ot que par l'utilisation
de {\tt \#undef} puisque certaines options de compilation sont pass\'ees depuis la ligne de commande au compilateur
par l'option {\tt -D} et risquent d'\^etre d\'esactiv\'ees si un {\tt \#undef} est utilis\'e.

\section{Simulateur sur PC}

Le simulateur d'interrogateur sur PC est un outil pour d\'everminer les problemes d'affichage (d\'epassement de
capacit\'e) et d'immunit\'e au bruit. Le fichier {\tt dummy\_pc.c} contient les fonctions qui \'emulent
les r\'eponses des composants mat\'eriels des interrogateurs. En tant que tel, un double-r\'esonateur est simul\'e (sous forme de
r\'eponse somme de deux gaussiennes) dans la fontion {\tt interroge()} qui renvoie une information de puissance en 
fonction de la fr\'equence \'emise par le DDS. Cet outil semble particuli\`erement appropri\'e pour simuler une mesure
bruit\'ee, pour le moment avec une contribution de bruit blanc sur l'amplitude mais qui pourrait \^etre enrichie de
diverses statistiques de bruit si besoin est.

Le simulateur doit \^etre ex\'ecut\'e depuis une console en mode texte sous GNU/Linux puisque la gestion du clavier
ne semble pas appropri\'ee pour l'environnement graphique X11.

\end{document}
